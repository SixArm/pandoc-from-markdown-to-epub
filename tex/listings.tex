\usepackage{listings}
\usepackage{xcolor}

\lstset{
    backgroundcolor=\color[RGB]{240,240,240},
    basicstyle=\footnotesize\ttfamily\linespread{1.1},  % the size of the fonts that are used for the code; posssible values are (\ttfamily, \footnotesize, etc.)
    breaklines=true,                % sets automatic line breaking
    breakatwhitespace=true,         % sets if automatic breaks should only happen at whitespace
    breakautoindent=true,
    breaklines=true,
    captionpos=b,
    commentstyle=\color[rgb]{0.56,0.35,0.01}\itshape,
    escapeinside={\%*}{*)},
    frame=single,	                 % adds a frame around the code
    framesep=8pt,
    keywordstyle=\color[rgb]{0.13,0.29,0.53}\bfseries,
    linewidth=\textwidth,
    numbers=none,                    % where to put the line-numbers; possible values are (none, left, right)
    numbersep=5pt,                   % how far the line-numbers are from the code
    numberstyle=\tiny\color{gray},   % the style that is used for the line-numbers
    rulecolor=\color[RGB]{220,220,220}, % the frame color; we prefer slightly-darker than the background color
    showspaces=false,                % show spaces everywhere adding particular underscores; it overrides 'showstringspaces'
    showstringspaces=false,          % underline spaces within strings only
    showtabs=false,                  % show tabs within strings adding particular underscores
    stepnumber=2,                    % the step between two line-numbers. If it's 1, each line will be numbered
    stringstyle=\color[rgb]{0.31,0.60,0.02},
    tabsize=4,                       % sets default tabsize to 2 spaces
    xleftmargin=8pt,                 % use the same value as framesep
    xrightmargin=8pt,                % use the same value as framesep
}

% Unused:
%   basewidth=0.9em,

% \lstset{ 
%   backgroundcolor=\color{white},   % choose the background color; you must add \usepackage{color} or \usepackage{xcolor}; should come as last argument
%   captionpos=b,                    % sets the caption-position to bottom
%   commentstyle=\color{mygreen},    % comment style
%   deletekeywords={...},            % if you want to delete keywords from the given language
%   escapeinside={\%*}{*)},          % if you want to add LaTeX within your code
%   extendedchars=true,              % lets you use non-ASCII characters; for 8-bits encodings only, does not work with UTF-8
%   firstnumber=1000,                % start line enumeration with line 1000
%   keepspaces=true,                 % keeps spaces in text, useful for keeping indentation of code (possibly needs columns=flexible)
%   keywordstyle=\color{blue},       % keyword style
%   language=Octave,                 % the language of the code
%   morekeywords={*,...},            % if you want to add more keywords to the set
%   rulecolor=\color{black},         % if not set, the frame-color may be changed on line-breaks within not-black text (e.g. comments (green here))
%   stringstyle=\color{mymauve},     % string literal style
%   title=\lstname                   % show the filename of files included with \lstinputlisting; also try caption instead of title
% }
